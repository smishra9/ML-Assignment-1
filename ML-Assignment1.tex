% Options for packages loaded elsewhere
\PassOptionsToPackage{unicode}{hyperref}
\PassOptionsToPackage{hyphens}{url}
%
\documentclass[
]{article}
\usepackage{lmodern}
\usepackage{amssymb,amsmath}
\usepackage{ifxetex,ifluatex}
\ifnum 0\ifxetex 1\fi\ifluatex 1\fi=0 % if pdftex
  \usepackage[T1]{fontenc}
  \usepackage[utf8]{inputenc}
  \usepackage{textcomp} % provide euro and other symbols
\else % if luatex or xetex
  \usepackage{unicode-math}
  \defaultfontfeatures{Scale=MatchLowercase}
  \defaultfontfeatures[\rmfamily]{Ligatures=TeX,Scale=1}
\fi
% Use upquote if available, for straight quotes in verbatim environments
\IfFileExists{upquote.sty}{\usepackage{upquote}}{}
\IfFileExists{microtype.sty}{% use microtype if available
  \usepackage[]{microtype}
  \UseMicrotypeSet[protrusion]{basicmath} % disable protrusion for tt fonts
}{}
\makeatletter
\@ifundefined{KOMAClassName}{% if non-KOMA class
  \IfFileExists{parskip.sty}{%
    \usepackage{parskip}
  }{% else
    \setlength{\parindent}{0pt}
    \setlength{\parskip}{6pt plus 2pt minus 1pt}}
}{% if KOMA class
  \KOMAoptions{parskip=half}}
\makeatother
\usepackage{xcolor}
\IfFileExists{xurl.sty}{\usepackage{xurl}}{} % add URL line breaks if available
\IfFileExists{bookmark.sty}{\usepackage{bookmark}}{\usepackage{hyperref}}
\hypersetup{
  pdftitle={ML Assignment 1},
  pdfauthor={Sumit Dutt Mishra},
  hidelinks,
  pdfcreator={LaTeX via pandoc}}
\urlstyle{same} % disable monospaced font for URLs
\usepackage[margin=1in]{geometry}
\usepackage{color}
\usepackage{fancyvrb}
\newcommand{\VerbBar}{|}
\newcommand{\VERB}{\Verb[commandchars=\\\{\}]}
\DefineVerbatimEnvironment{Highlighting}{Verbatim}{commandchars=\\\{\}}
% Add ',fontsize=\small' for more characters per line
\usepackage{framed}
\definecolor{shadecolor}{RGB}{248,248,248}
\newenvironment{Shaded}{\begin{snugshade}}{\end{snugshade}}
\newcommand{\AlertTok}[1]{\textcolor[rgb]{0.94,0.16,0.16}{#1}}
\newcommand{\AnnotationTok}[1]{\textcolor[rgb]{0.56,0.35,0.01}{\textbf{\textit{#1}}}}
\newcommand{\AttributeTok}[1]{\textcolor[rgb]{0.77,0.63,0.00}{#1}}
\newcommand{\BaseNTok}[1]{\textcolor[rgb]{0.00,0.00,0.81}{#1}}
\newcommand{\BuiltInTok}[1]{#1}
\newcommand{\CharTok}[1]{\textcolor[rgb]{0.31,0.60,0.02}{#1}}
\newcommand{\CommentTok}[1]{\textcolor[rgb]{0.56,0.35,0.01}{\textit{#1}}}
\newcommand{\CommentVarTok}[1]{\textcolor[rgb]{0.56,0.35,0.01}{\textbf{\textit{#1}}}}
\newcommand{\ConstantTok}[1]{\textcolor[rgb]{0.00,0.00,0.00}{#1}}
\newcommand{\ControlFlowTok}[1]{\textcolor[rgb]{0.13,0.29,0.53}{\textbf{#1}}}
\newcommand{\DataTypeTok}[1]{\textcolor[rgb]{0.13,0.29,0.53}{#1}}
\newcommand{\DecValTok}[1]{\textcolor[rgb]{0.00,0.00,0.81}{#1}}
\newcommand{\DocumentationTok}[1]{\textcolor[rgb]{0.56,0.35,0.01}{\textbf{\textit{#1}}}}
\newcommand{\ErrorTok}[1]{\textcolor[rgb]{0.64,0.00,0.00}{\textbf{#1}}}
\newcommand{\ExtensionTok}[1]{#1}
\newcommand{\FloatTok}[1]{\textcolor[rgb]{0.00,0.00,0.81}{#1}}
\newcommand{\FunctionTok}[1]{\textcolor[rgb]{0.00,0.00,0.00}{#1}}
\newcommand{\ImportTok}[1]{#1}
\newcommand{\InformationTok}[1]{\textcolor[rgb]{0.56,0.35,0.01}{\textbf{\textit{#1}}}}
\newcommand{\KeywordTok}[1]{\textcolor[rgb]{0.13,0.29,0.53}{\textbf{#1}}}
\newcommand{\NormalTok}[1]{#1}
\newcommand{\OperatorTok}[1]{\textcolor[rgb]{0.81,0.36,0.00}{\textbf{#1}}}
\newcommand{\OtherTok}[1]{\textcolor[rgb]{0.56,0.35,0.01}{#1}}
\newcommand{\PreprocessorTok}[1]{\textcolor[rgb]{0.56,0.35,0.01}{\textit{#1}}}
\newcommand{\RegionMarkerTok}[1]{#1}
\newcommand{\SpecialCharTok}[1]{\textcolor[rgb]{0.00,0.00,0.00}{#1}}
\newcommand{\SpecialStringTok}[1]{\textcolor[rgb]{0.31,0.60,0.02}{#1}}
\newcommand{\StringTok}[1]{\textcolor[rgb]{0.31,0.60,0.02}{#1}}
\newcommand{\VariableTok}[1]{\textcolor[rgb]{0.00,0.00,0.00}{#1}}
\newcommand{\VerbatimStringTok}[1]{\textcolor[rgb]{0.31,0.60,0.02}{#1}}
\newcommand{\WarningTok}[1]{\textcolor[rgb]{0.56,0.35,0.01}{\textbf{\textit{#1}}}}
\usepackage{graphicx}
\makeatletter
\def\maxwidth{\ifdim\Gin@nat@width>\linewidth\linewidth\else\Gin@nat@width\fi}
\def\maxheight{\ifdim\Gin@nat@height>\textheight\textheight\else\Gin@nat@height\fi}
\makeatother
% Scale images if necessary, so that they will not overflow the page
% margins by default, and it is still possible to overwrite the defaults
% using explicit options in \includegraphics[width, height, ...]{}
\setkeys{Gin}{width=\maxwidth,height=\maxheight,keepaspectratio}
% Set default figure placement to htbp
\makeatletter
\def\fps@figure{htbp}
\makeatother
\setlength{\emergencystretch}{3em} % prevent overfull lines
\providecommand{\tightlist}{%
  \setlength{\itemsep}{0pt}\setlength{\parskip}{0pt}}
\setcounter{secnumdepth}{-\maxdimen} % remove section numbering

\title{ML Assignment 1}
\author{Sumit Dutt Mishra}
\date{1/31/2021}

\begin{document}
\maketitle

\hypertarget{q1-data-source}{%
\paragraph{Q1: Data Source}\label{q1-data-source}}

Our dataset \textbf{Accounting Professionals} has been extarcted from
\textbf{KAGGLE}, is a collection of databases and datasets used by
machine learning community for the analysis of ML algorithms.

\hypertarget{q2-importing-accounting-professionals-data-set-to-r-studio}{%
\paragraph{Q2: Importing Accounting Professionals data set to R
Studio}\label{q2-importing-accounting-professionals-data-set-to-r-studio}}

\begin{Shaded}
\begin{Highlighting}[]
\KeywordTok{setwd}\NormalTok{(}\StringTok{"C:}\CharTok{\textbackslash{}\textbackslash{}}\StringTok{Users}\CharTok{\textbackslash{}\textbackslash{}}\StringTok{prerak}\CharTok{\textbackslash{}\textbackslash{}}\StringTok{Desktop}\CharTok{\textbackslash{}\textbackslash{}}\StringTok{sumit"}\NormalTok{)}
\NormalTok{acc\_prof \textless{}{-}}\StringTok{ }\KeywordTok{read.csv}\NormalTok{(}\StringTok{"Accounting Professionals.csv"}\NormalTok{)}
\KeywordTok{head}\NormalTok{(acc\_prof)}
\end{Highlighting}
\end{Shaded}

\begin{verbatim}
##   Employee Gender Years.of.Service Years.Undergraduate.Study Graduate.Degree.
## 1        1      F               17                         4                N
## 2        2      F                6                         2                N
## 3        3      M                8                         4                Y
## 4        4      F                8                         4                Y
## 5        5      M               16                         4                Y
## 6        6      F               21                         1                N
##   CPA. Age.Group
## 1    Y     41-45
## 2    N     26-30
## 3    Y     31-35
## 4    N     31-35
## 5    Y     36-40
## 6    Y     51-55
\end{verbatim}

\hypertarget{q3descriptive-statistics-of-acc.prof-data-set}{%
\paragraph{Q3:Descriptive Statistics of acc.prof data
set}\label{q3descriptive-statistics-of-acc.prof-data-set}}

Inspecting DataSet

\begin{Shaded}
\begin{Highlighting}[]
\KeywordTok{summary}\NormalTok{(acc\_prof)}
\end{Highlighting}
\end{Shaded}

\begin{verbatim}
##     Employee       Gender          Years.of.Service Years.Undergraduate.Study
##  Min.   : 1.0   Length:27          Min.   : 5.0     Min.   :0.00             
##  1st Qu.: 7.5   Class :character   1st Qu.: 8.0     1st Qu.:3.00             
##  Median :14.0   Mode  :character   Median :10.0     Median :4.00             
##  Mean   :14.0                      Mean   :14.7     Mean   :3.37             
##  3rd Qu.:20.5                      3rd Qu.:20.5     3rd Qu.:4.00             
##  Max.   :27.0                      Max.   :31.0     Max.   :4.00             
##  Graduate.Degree.       CPA.            Age.Group        
##  Length:27          Length:27          Length:27         
##  Class :character   Class :character   Class :character  
##  Mode  :character   Mode  :character   Mode  :character  
##                                                          
##                                                          
## 
\end{verbatim}

\textbf{Arithmetic Mean:} AM of a set of observation is defined as their
sum divided number of observations.

\begin{Shaded}
\begin{Highlighting}[]
\CommentTok{\# Average Number of Years of Service}
\KeywordTok{mean}\NormalTok{(acc\_prof}\OperatorTok{$}\NormalTok{Years.of.Service)}
\end{Highlighting}
\end{Shaded}

\begin{verbatim}
## [1] 14.7037
\end{verbatim}

\textbf{Median:} is the middle value of the observation.

\begin{Shaded}
\begin{Highlighting}[]
\CommentTok{\# Median of Age Group}
\KeywordTok{median}\NormalTok{(}\KeywordTok{sort}\NormalTok{(acc\_prof}\OperatorTok{$}\NormalTok{Age.Group))}
\end{Highlighting}
\end{Shaded}

\begin{verbatim}
## [1] "36-40"
\end{verbatim}

\textbf{Mode:} it refers to the value which occurs to the maximum
frequency.

\begin{Shaded}
\begin{Highlighting}[]
\KeywordTok{sort}\NormalTok{(}\KeywordTok{table}\NormalTok{(acc\_prof}\OperatorTok{$}\NormalTok{Gender), }\DataTypeTok{decreasing =} \OtherTok{TRUE}\NormalTok{)}
\end{Highlighting}
\end{Shaded}

\begin{verbatim}
## 
##  F  M 
## 14 13
\end{verbatim}

\hypertarget{q3-transformation}{%
\paragraph{Q3: Transformation}\label{q3-transformation}}

\begin{Shaded}
\begin{Highlighting}[]
\CommentTok{\# Applying log transformation on years of service column}
\NormalTok{log\_trans \textless{}{-}}\StringTok{ }\KeywordTok{log10}\NormalTok{(acc\_prof}\OperatorTok{$}\NormalTok{Years.of.Service)}
\CommentTok{\#displaying first 20 values}
\NormalTok{log\_trans[}\DecValTok{1}\OperatorTok{:}\DecValTok{20}\NormalTok{]}
\end{Highlighting}
\end{Shaded}

\begin{verbatim}
##  [1] 1.2304489 0.7781513 0.9030900 0.9030900 1.2041200 1.3222193 1.4313638
##  [8] 0.8450980 0.9030900 1.3617278 0.9542425 0.9030900 0.9030900 1.4149733
## [15] 0.9542425 0.9542425 1.2787536 0.6989700 1.2787536 1.3010300
\end{verbatim}

\hypertarget{q4-visualisation}{%
\paragraph{Q4: Visualisation}\label{q4-visualisation}}

\textbf{Histogram Plot}

\begin{Shaded}
\begin{Highlighting}[]
\KeywordTok{library}\NormalTok{(ggplot2)}
\KeywordTok{ggplot}\NormalTok{(acc\_prof, }\KeywordTok{aes}\NormalTok{(}\DataTypeTok{x =}\NormalTok{ Years.Undergraduate.Study)) }\OperatorTok{+}
\StringTok{  }\KeywordTok{geom\_histogram}\NormalTok{() }\OperatorTok{+}\StringTok{ }
\StringTok{  }\KeywordTok{ggtitle}\NormalTok{(}\StringTok{\textquotesingle{}Histogram \textquotesingle{}}\NormalTok{)}
\end{Highlighting}
\end{Shaded}

\includegraphics{ML-Assignment1_files/figure-latex/unnamed-chunk-7-1.pdf}

\textbf{Scatter Plot}

\begin{Shaded}
\begin{Highlighting}[]
\KeywordTok{ggplot}\NormalTok{(acc\_prof, }\KeywordTok{aes}\NormalTok{(}\DataTypeTok{x=}\NormalTok{Years.of.Service, }\DataTypeTok{y=}\NormalTok{Years.Undergraduate.Study)) }\OperatorTok{+}\StringTok{ }
\StringTok{  }\KeywordTok{geom\_point}\NormalTok{() }\OperatorTok{+}
\StringTok{  }\KeywordTok{ggtitle}\NormalTok{(}\StringTok{\textquotesingle{}Scatter plot\textquotesingle{}}\NormalTok{)}
\end{Highlighting}
\end{Shaded}

\includegraphics{ML-Assignment1_files/figure-latex/unnamed-chunk-8-1.pdf}

Note that the \texttt{echo\ =\ FALSE} parameter was added to the code
chunk to prevent printing of the R code that generated the plot.

\end{document}
